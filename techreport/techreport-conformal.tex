\documentclass[11pt]{article}
%\documentclass[manuscript]{biometrika}
%\documentclass{biometrika}
%\documentclass[lineno]{biometrika}

\usepackage{amscd}
\usepackage{amssymb}
\usepackage{amsthm}
\usepackage{amsmath}
\usepackage{amsfonts}
\usepackage{natbib}
\usepackage{url}
\usepackage{graphicx,times}
\usepackage{epsfig}
\usepackage{mathtools}
\usepackage{xcolor}
%\usepackage{tikz-cd}
%\usepackage{setspace}
%\usepackage{epstopdf}
\usepackage{dsfont} % blackboard numerals

%\epstopdfDeclareGraphicsRule{.gif}{png}{.png}{convert gif:#1 png:\OutputFile}
%\AppendGraphicsExtensions{.gif}

\usepackage{geometry}
\geometry{margin=1in}

\newcommand{\R}{\mathbb{R}}
\newcommand{\Prob}{\mathbb{P}}
\newcommand{\exreal}{\overline{\R}}
\newcommand{\fatdot}{\,\cdot\,}
\newcommand{\Estar}{E^{\textstyle{*}}}
\newcommand{\Holder}{\mathcal{G}}
\newcommand{\Copt}{C^{(\alpha)}}
\newcommand{\Copthat}{\widehat{C}^{(\alpha)}}
\newcommand{\Coptloc}{\widehat{C}^{(\alpha)}_{\text{loc}}}
\newcommand{\Coptglm}{\widehat{C}^{(\alpha)}_{n, k}}
\newcommand{\Coptglmn}{C^{(\alpha)}_{n, k}}
\newcommand{\ptrue}{p_{\beta, \phi}}
\newcommand{\betahat}{\hat{\beta}}
\newcommand{\thetahat}{\hat{\theta}}
\newcommand{\phihat}{\hat{\phi}}
\newcommand{\psihat}{\hat{\psi}}
\newcommand{\phat}{p_{\betahat, \phihat}}
\newcommand{\phatnon}{\hat{p}_{\beta, \phi}}
\newcommand{\E}{\mathbb{E}}
\newcommand{\A}{\mathcal{A}}
\newcommand{\X}{\mathcal{X}}
\newcommand{\Y}{\mathcal{Y}}
\newcommand{\talpha}{t^{(\alpha)}}
\newcommand{\Aalpha}{A^{(\alpha)}}
\newcommand{\Aaxe}{A^{(\alpha)}_{x,\varepsilon}}
\newcommand{\thatalpha}{\hat{t}^{(\alpha)}}
\newcommand{\ttildealpha}{\tilde{t}^{(\alpha)}}
\newcommand{\tinfk}{t^{(\alpha)}_{\text{inf}, k}}
\newcommand{\tsupk}{t^{(\alpha)}_{\text{sup}, k}}
\newcommand{\rootn}{\sqrt{n}}

\newcommand{\xstar}{x^{\textstyle{*}}}
\newcommand{\zstar}{z^{\textstyle{*}}}
\newcommand{\Astar}{A^{\textstyle{*}}}

%\newcommand{\comment}[1]{\textcolor{red}{[#1]}}
\newcommand{\inner}[1]{\langle #1 \rangle}
\newcommand{\indicator}[1]{\mathds{1}\left\{ #1 \right\}}

\DeclarePairedDelimiter\abs{\lvert}{\rvert}
\DeclarePairedDelimiter\norm{\lVert}{\rVert}
\DeclarePairedDelimiter\ceil{\lceil}{\rceil}
\DeclarePairedDelimiter\floor{\lfloor}{\rfloor}
\DeclareMathOperator{\Var}{Var}
\DeclareMathOperator{\dom}{dom}

\newtheorem{lem}{Lemma} 
\newtheorem{thm}{Theorem}
\newtheorem{cor}{Corollary} 
\newtheorem{defn}{Definition} 
\newtheorem{prop}{Proposition} 

\allowdisplaybreaks

\title{Technical report for ``Conformal prediction for exponential 
  families and generalized linear models''}
\author{Daniel J. Eck}


\begin{document}

\maketitle

\begin{abstract}
In this supplementary materials document we provide all of the code to 
reproduce the analyses which appear in the paper \citet{eck2019conformal} and 
the R package \citet{eck2018conformalR}.  Several additional analyses are 
provided which illustrate the advantages of conformal prediction regions.  In 
particular, we provide evidence that the parametric conformal prediction 
region, developed in \citet{eck2019conformal}, compares favorably to other 
prediction regions even when the parametric model is misspecified.  
\end{abstract}

\vspace*{0.5cm}

\tableofcontents

\vspace*{0.5cm}

The following R packages are required in order to replicate the calculations 
within this document. 

\begin{knitrout}
\definecolor{shadecolor}{rgb}{0.969, 0.969, 0.969}\color{fgcolor}\begin{kframe}
\begin{alltt}
\hlkwd{library}\hlstd{(parallel)}
\hlkwd{library}\hlstd{(MASS)}
\hlkwd{library}\hlstd{(statmod)}
\hlkwd{library}\hlstd{(HDInterval)}
\hlkwd{library}\hlstd{(conformal.glm)} \hlcom{## https://github.com/DEck13/conformal.glm}
\hlkwd{library}\hlstd{(conformalInference)} \hlcom{## https://github.com/ryantibs/conformal}
\hlkwd{library}\hlstd{(faraway)}
\hlkwd{library}\hlstd{(geometry)}
\hlkwd{library}\hlstd{(xtable)}
\hlkwd{library}\hlstd{(rgl)}
\end{alltt}
\end{kframe}
\end{knitrout}

We set the error tolerance to be $\alpha = 0.10$ for all prediction regions 
unless otherwise noted.


\section{Introduction/Summary of simulation results}

This manuscript (and corresponding .Rnw file) provides all of the code to 
reproduce the analyses which appear in the paper \citet{eck2019conformal}, the 
README file in the R package \citet{eck2018conformalR}, and this document.  
Several additional analyses to those presented in 
\citet{eck2019conformal} are provided in this document.  
Of particular intherest, we investigate the performance of parametric 
conformal prediction regions under model misspecification.  
Specifically, we focus on settings where the underlying data is generated 
via a Gamma distribution and parametric prediction regions are obtained using a 
cubic regression model assuming homoscedastic normal errors.  The cubic fit is 
chosen because it is intuitive and it fits the Gamma data better than a simple 
linear regression model or a quadratic model.  

Our goal in this manuscript is to demonstrate the advantages and disadvantages 
of our binned and transformation based parametric conformal prediction regions 
\citep{eck2019conformal} 
compared with the nonparametric conformal prediction region 
\citep{lei2014distribution}, the least squares (LS) conformal prediction 
region \citep{lei2018distribution} obtained from conformalized residual scores, 
the least squares locally weighted (LSLW) conformal prediction region 
\citep[Section 5.2]{lei2018distribution} obtained from conformalized locally 
weighted residual scores, and the highest density (HD) region. In analyses 
with model misspecification, the parametric, LS, and LSLW conformal prediction 
regions and HD prediction region are constructed under the misspecified 
model.  The binning used to construct the parametric and 
nonparametric conformal prediction regions follows the bin width asymptotics 
of \citet{lei2014distribution}.

%In our simulations 
We find that the binned parametric conformal prediction 
region performs well even when the model is misspecified.  By construction, 
this region, along with the nonparametric conformal prediction region, 
maintains finite-sample local validity with respect to binning.  
These conformal prediction regions therefore achieves finite-sample marginal 
validity.  The guarantee of finite-sample marginal and local validity are 
noted benefits of these conformal prediction regions 
\citep{lei2014distribution, eck2019conformal}.  
However, the binned parametric and nonparametric conformal prediction 
regions are visually very different, as seen in 
Section~\ref{sec:plotsofregions}, and give different prediction errors at 
small to moderate sample sizes.  We see that the binned parametric conformal 
prediction region adapts naturally to the data when the model is correctly 
specified or modest deviations from the specified model are present.  
On the other hand, the nonparametric conformal prediction region does not 
adapt well to data obtained from a Gamma regression model or data obtained 
from a linear regression model with a steep mean function where steepness is 
relative to the variability about the mean function.

We find that the transformation based parametric conformal prediction region 
performs similarily to the HD prediction region under the assumed model when 
the assumed model is either the data generating model or only modest 
departures exist between the assumed model and the data generating model.  
This parametric conformal prediction region can be prohibitively large and 
inapproriate when the assumed model is in strong disagreement with the data 
generating model.  However, this parametric conformal prediction region 
maintains finite sample marginal coverage in such settings.

The LS conformal prediction region obtains marginal validity 
\citep{lei2018distribution} but performs poorly when deviations about the 
estimated mean function are either not symmetric, not constant, or both.
When heterogeneity is present, the LS conformal prediction region exhibits 
undercoverage in regions where variability about the mean function is large 
and overcoverage in regions where variability about the mean function is 
small.  This conformal prediction region is very sensitive to model 
misspecification.  
The LSLW conformal prediction region also obtains marginal validity 
\citep[Section 5.2]{lei2018distribution} and it is far less sensitive to 
model misspecification than the LS conformal prediction region.  
However, the LSLW conformal prediction region is not appropriate 
when deviations about an estimated mean function are obviously not 
symmetric, as evidenced in Section~\ref{sec:gammaplots}.



\section{Illustrative example from the README file}
\label{sec:README}

We provide a gamma regression example with perfect model specification to 
illustrate the performance of conformal predictions when the model is known 
and the model does not have additive symmetric errors.  We also compare 
conformal prediction regions to the oracle highest density region under 
the correct model. This example is included in the corresponding paper  
\citet{eck2019conformal} and the README file of the corresponding 
R package \texttt{conformal.glm} \citep{eck2018conformalR}.




























































































































































































































































































































































































